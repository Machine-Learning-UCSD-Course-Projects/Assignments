\documentclass{acm_proc_article-sp}
\begin{document}

\title{Punctuation Prediction using Conditional Random Fields}

\numberofauthors{3} %  
\author{
\alignauthor
% 1st. author
Suvir Jain,\\
\affaddr{Computer Science Department}\\
\affaddr{University of California, San Diego}\\
\email{suj011@cs.ucsd.edu}
% 2nd. author
\alignauthor
Gaurav Saxena\\
\affaddr{Computer Science Department}\\
\affaddr{University of California, San Diego}\\
\email{gsaxena@cs.ucsd.edu}
%3rd Author%
\alignauthor
Kashyap Tumkur\\
\affaddr{Computer Science Department}\\
\affaddr{University of California, San Diego}\\
\email{gsaxena@cs.ucsd.edu}
}

\maketitle
\begin{abstract}
\end{abstract}

\section{Introduction}
\section{Preliminaries}
\subsection{Assumptions}
Tag: SPACE to show No punctuation
Tag START to mark the beginning of the tag sequence
Tag: PERIOD, QUESTION MARK to mark the end of the tag sequence

\section{Implementation}
\subsection{Algorithms for CRF}
\subsubsection*{Viterbi path algorithm for Inference}
\paragraph*{Proof of correctness}
\subsubsection*{Forward and Backward Vectors}
\paragraph*{Proof of correctness}
Find Z using both $\alpha$ and $\beta$ vectors
\subsubsection{Learning using SGD}
\paragraph{Proof of correctness}
Prove derivatives are correct
\subsection{Learning using Collins Perceptron}
\paragraph{Proof of correctness}
\subsubsection{Feature Functions}
\begin{enumerate}
\item -ing words
\item interrogative words, subject-verb inversion
\item connectives like however
\item conjunctions
\item Interjections
\end{enumerate}
\subsection{Data cleaning}
\subsection{Overfitting}
\subsubsection{Validation}
\subsubsection{Regularization?}
\subsubsection{Feature Scaling?}
\subsubsection{Randomization}
Can we have our feature functions output only values between 0 and 1
\section{Experiments}
\subsection{Experiment 1: Learning using SGD}
Charts proving convergence of SGD
Charts showing accuracy on training, validation and test set
\subsection{Experiment 2: Learning using Collins Perceptron}
Charts proving convergence of CP
Charts showing accuracy on training, validation and test set
\section{Optimizations}
Optimizations done to the algorithm to improve running time like:
\begin{enumerate}
\item Run the algorithms on a random smaller subset to speed up implementation
\item Use MATLAB profiler to find bottlenecks and remove them
\end{enumerate}
\section{Conclusion}
\subsection{Lessons Learnt}
\subsection{Discussion}
Comparison of two approaches in terms of time to converge, their accuracies on training, validation and test sets
%
% The following two commands are all you need in the
% initial runs of your .tex file to
% produce the bibliography for the citations in your paper.
\bibliographystyle{abbrv}
\bibliography{Report} 
\end{document}
